\chapter{Предисловие}

Это руководство расскажет вам о самом крутом веб-сервере на планете: Mongrel2.
Оно написано для людей с чувством юмора, которые нацелены на достижение результата
в работе. Если ваша деятельность связана с эксплуатацией серверов, вы --- разработчик,
хакер или просто любопытствующий, то он (сервер) для вас. Замечу, если вы слишком серьезны
и считаете, что такой ``простоватый'' стиль изложения (A.K.A.\ хороший, занимательный стиль)
неуместен в мануалах, идите читайте исходные коды и не напрягайте окружающих.

Если вы ещё не поняли, то эта книга будет весёлой и слегка откровенной; не с целью обидеть
вас, уважаемый читатель, а просто чтобы не дать вам заснуть в процессе чтения.

\section*{Типографика}

Людей в вебе обычно можно разделить на три основных типа: Стивы, Дейкстры и Кнуты.

Стивы думают, что весь интернет должен быть миром самых приятных ощущений,
где все странички созданы с точностью до пикселя, со всевозможными тенями, полутенями
и градиентами, и прочими плюшками. Для них дизайн стоит во главе угла и стабильность
работы сайта не так уж важна, если она не противоречит графическому дизайну.
Стивы от интернета думают, что Дейкстры от интернета уничтожают вселенную такими
вещами как ``функциональность'', ``безопасность'' и ``стабильность''. Они и рассуждают
как реальные Стивы Джобсы --- пусть уж всё выглядит фантастически, а всякие технические
огрехи мы скроем за занавесом самого продвинутого маркетинга.

Дейкстры воспринимают интернет как абсолютно небезопасное место. И до тех пор пока
он не будет ригулироваться с помощью тщательно разработанных и коллективно одобренных
научных трудов, он будет оставаться некоей зловонной сливной ямой. Для Дейкстр мир опасен,
и только параноидальная одержимость безопасностью и стабильностью может улучшить
ситуацию. Они хотят, чтобы любой софт отвергал реальность и был разработан с
исключительно математических позиций. И их напрягает тот факт, что Стивы суетятся
вокруг, пытаясь украсить мир какими-то бесполезными красками, словами и вещами,
которые ведут к неясности и счастью.

Типографика в этой книге и во всём проекте --- для Кнутов этого мира. Мне нравится
думать о Кнутах, как о практичных и, тем не менее, профессиональных людях с адекватным
чувством юмора. Они --- те, кто выполняют работу, балансируя между выдающейся типографикой
и надёжной функциональностью. Они не фанатики, а практичные, прямолинейные люди.

Вот почему эта книга свёрстана с помощью пакета \TeX и использует те же шрифты, что и \TeX.
